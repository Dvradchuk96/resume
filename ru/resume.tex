%%%%%%%%%%%%%%%%%%%%%%%%%%%%%%%%%%%%%%%%%
% Medium Length Professional CV
% LaTeX Template
% Version 2.0 (8/5/13)
%
% This template has been downloaded from:
% http://www.LaTeXTemplates.com
%
% Original author:
% Trey Hunner (http://www.treyhunner.com/)
%
% Important note:
% This template requires the resume.cls file to be in the same directory as the
% .tex file. The resume.cls file provides the resume style used for structuring the
% document.
%
%%%%%%%%%%%%%%%%%%%%%%%%%%%%%%%%%%%%%%%%%

%----------------------------------------------------------------------------------------
%	PACKAGES AND OTHER DOCUMENT CONFIGURATIONS
%----------------------------------------------------------------------------------------

\documentclass{resume} % Use the custom resume.cls style

\usepackage[left=0.75in,top=0.6in,right=0.75in,bottom=0.6in]{geometry} % Document margins

\name{Данил Радчук} % Your name
\address{+7(960)-882-76-82 \\ dvradchuk96@gmail.com} % Your phone number and email
\address{https://www.linkedin.com/in/danil-radchuk \\ https://github.com/danradchuk} % Your LinkedIn and GitHub

\begin{document}

%----------------------------------------------------------------------------------------
%	OBJECTIVE SECTION
%----------------------------------------------------------------------------------------

\begin{rSection}{Цель}

    В поисках позиции, которая будет соответствовать моим навыкам, а также принесет новые возможности для развития.

\end{rSection}

%----------------------------------------------------------------------------------------
%	EDUCATION SECTION
%----------------------------------------------------------------------------------------

\begin{rSection}{Образование}

{\bf Волгоградский Государственный Технический Университет} \hfill {\em Сентябрь 2014 - Июль 2018} \\ 
Бакалавр прикладной информатики

\end{rSection}

%----------------------------------------------------------------------------------------
%	TECHNICAL SKILLS SECTION
%----------------------------------------------------------------------------------------

\begin{rSection}{Навыки}

    \begin{tabular}{ @{} >{\bfseries}l @{\hspace{6ex}} l }

    Фреймворки и библиотеки & Spring, Spring Boot, Spring WebFlux, Testcontainers \\ 
    Технологии & Docker, Linux, MongoDB, PostgreSQL, RabbitMQ, 
    \\ &  Kafka, Git, Grafana, Prometheus \\ 
    Языки & Свободно: Java, Kotlin, SQL. Знаком: Go, Scala, Haskell, Python
    
    \end{tabular}
    
\end{rSection}

%----------------------------------------------------------------------------------------
%	WORK EXPERIENCE SECTION
%----------------------------------------------------------------------------------------

\begin{rSection}{Опыт}

\begin{rSubsection}{Приложение Кошелёк}{Декабрь 2020 - По настоящее время}{Software Engineer}{Удаленно, Россия}
\item Разработка и поддержка микросервисов с использованием Spring, Spring Boot.
\item Разработка и поддержка API, используемое более чем 150+ ритейлерами.
\item Разработка и поддержка кастомных интеграций с партнерами.(Магнит, Пятерочка и др.)
\item Разработал библиотеку для работы с TOTP (алгоритм для создания одноразовых паролей).
\item Перевел сервисы команды на новый стек (Java 11, Spring Boot).
\end{rSubsection}

%------------------------------------------------

\begin{rSubsection}{Waveaccess}{Июль 2019 - Ноябрь 2020}{Software Engineer}{Удаленно, Россия}
\item Проектирование и разработка микросервисов с использованием Spring Boot и Project Reactor.
\item Внедрил метрики в проект (Grafana и Prometheus).
\end{rSubsection}

%------------------------------------------------

\begin{rSubsection}{Unitbean}{Сентябрь 2017 - Июнь 2019}{Full-stack Software Engineer}{Волгоград, Россия}
\item Разработка и поддержка web \& mobile API для food-tech приложения.
\item Разработка и поддержка фронтенд части приложения с использованием jQuery и JSP.
\item Участие во внутренних tech-talks компании.
\end{rSubsection}

\end{rSection}

%----------------------------------------------------------------------------------------
%	EXAMPLE SECTION
%----------------------------------------------------------------------------------------

%\begin{rSection}{Section Name}

%Section content\ldots

%\end{rSection}

%----------------------------------------------------------------------------------------

\end{document}
